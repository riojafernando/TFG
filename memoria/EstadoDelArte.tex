\chapter{Estado del Arte}

	En este capítulo se va a introducir el interés por analizar y procesar señales médicas así como la evolución de las bases de datos hasta la actualidad.\par 
\section{Contexto del estudio}
	El procesamiento digital de señales médicas ha ido muy ligado a la evolución de
las mejoras en los sistemas de computo.En este punto se puede ver cómo ha sido este
progreso en las últimas décadas:
\begin{itemize}
\item En 1952 Arthur Samuel escribe el primer programa de ordenador capaz de aprender. Consistia en un programa que jugaba a las damas y que mejoraba su juego después de cada partida.
\item Durante los años 1974 y 1980 tiene lugar el llamado AI Winter, ya que numerosas
empresas que financiaban los proyectos de inteligencia artificial, dejan de invertir
porque no se producen tantos avances para las expectativas creadas.
\item Los años 80 están marcados por la aparición de sistemas expertos basados en reglas y condiciones. Por ejemplo, en 1981 Gerald Dejong introduce el concepto “Explanation Based Learning” (EBL), donde un computador analiza datos de entrenamiento y crea reglas generales que le permiten descartar los datos menos importantes.
\item A finales de los 80 y durante buena parte de los 90, tiene lugar el segundo AI Winter. En esta ocasión los efectos duraron prácticamente hasta los 2000. Mencionar que en 1997, el ordenador Deep Blue, de IBM ganó al campeón del mundo de ajedrez Gary Kaspárov.
\item El aumento de la potencia y rapidez de cálculo junto con la gran abundancia de datos disponibles han vuelto a lanzar el campo de Machine Learning. Muchas empresas están transformando sus negocios hacia los datos y están incorporando técnicas de Machine Learning en sus procesos, productos y servicios para obtener ventajas sobre la competencia. También en el campo de la medicina, donde además de los diagnósticos del personal facultativo, se incorpora el análisis de datos puramente estadístico como ayuda para diagnosticar al paciente.
\end{itemize}

	En este contexto surge la plataforma Physionet, de donde se han obtenido los datos para este TFG.
\section{¿Qué es Physionet?}

	PhysioNet ofrece acceso gratuito a través de la web a grandes colecciones de señales fisiológicas registradas y software relacionado de código abierto. El sitio web de PhysioNet es un servicio público de PhysioNet Research Resource for Complex Physiologic Signals, financiado por el Instituto Nacional de Imágenes Biomédicas y Bioingeniería (NIBIB) y el  Instituto Nacional de Ciencias Médicas Generales (NIGMS) en los Institutos Nacionales de Salud.\par 
	PhysioNet Resource, creado en 1999, tiene como objetivo estimular la investiga-
ción actual y las nuevas exploraciones en el estudio de señales fisiológicas y biomédicas complejas. Tiene tres componentes estrechamente independientes:

\begin{itemize}
\item \textbf{Physiobank:} es un gran y creciente archivo de grabaciones digitales bien caracterizadas de señales fisiológicas, series de tiempo y datos relacionados para su uso por parte de la comunidad de investigación biomédica.
\item \textbf{PhysioToolKit:} es una biblioteca de software para procesamiento y análisis de señal fisiológica, detección de eventos fisiológicamente significativos utilizando técnicas clásicas y métodos novedosos, como estadística avanzada o dinámica no lineal.
\item \textbf{PhysioNetWorks:} es un laboratorio virtual donde se puede trabajar con colaboradores de cualquier parte del mundo para crear, evaluar, mejorar, documentar y preparar nuevos datos y trabajos de software para su publicación en PhysioNet.
\end{itemize}

	El sitio web de PhysioNet fue establecido por el Resource como su mecanismo para el intercambio libre y abierto de señales biomédicas grabadas y software de código abierto para analizarlos, proporcionando instalaciones para el análisis cooperativo de datos y la evaluación de nuevos algoritmos propuestos.\par 
	
	Además de proporcionar acceso electrónico gratuito a datos PhysioBank y software PhysioToolkit, y espacios de trabajo seguros para el desarrollo colaborativo de nuevos datos y software dentro de PhysioNetWorks, el sitio web PhysioNet ofrece servicio y capacitación a través de tutoriales en línea para ayudar a los usuarios en su inicio y para niveles más avanzados.\par 
	
	Physionet lanza competiciones (challenges) abiertas planteando un problema para que la comunidad aporte soluciones. El seleccionado para este TFG, corresponde al del año 2015 y tiene por título: \textit{Reducing False Arrhythmia Alarms in the ICU: the PhysioNet/Computing in Cardiology Challenge 2015}.
