\begin{thebibliography}{1}
	
	\bibitem{PhysionetIntro}
	PHYSIONET. Challenge 2015 Introduction,
	\url{https://physionet.org/challenge/2015/#introduction},
	
	\bibitem{Kaggle}
	KAGGLE. About Kaggle,
	\url{https://www.kaggle.com/kaggle},
	
	\bibitem{PCAWiki}
	WIKIPEDIA. PCA,
	\url{https://en.wikipedia.org/wiki/Principal_component_analysis},
	
	\bibitem{ecgwiki}
	WIKIPEDIA. Electrocardiograma,
	\url{https://es.wikipedia.org/wiki/Electrocardiograma},
	
	\bibitem{dibujo-ondap}
	MY EKG. Ondas del electrocargiograma,
	\url{http://www.my-ekg.com/generalidades-ekg/ondas-electrocardiograma.html},
	
	\bibitem{arritmia}
	MEDLINEPLUS. Enciclopedia Médica,
	\url{https://medlineplus.gov/spanish/ency/article/001101.htm},
	
	\bibitem{partes_challenge}
	PHYSIONET. Challengue 2015 Rules and Deadlines,
	\url{https://www.physionet.org/challenge/2015/#rules-and-deadlines},
	
	\bibitem{pleti}
	TURMERO, P. Mediciones fotopletismográficas,
	\url{http://www.monografias.com/trabajos104/mediciones-fotopletismograficas/mediciones-fotopletismograficas.shtml},
	
	\bibitem{respiracion}
	VALDERAS, MT. VALLVERDÚ, M. CAMINAL, P.
	«Extracción de la señal de respiración a partir del
	electrocardiograma», Actas de las XXXVI Jornadas
	de Automática, Bilbao, 2015,
	
	\bibitem{ISL}
	James, G., Witten, D., Hastie, T., and Tibshirani, R. 
	An Introduction to Statistical Learning: with Applications in R. Springer Science and Business Media, 2013.
	
	
	\bibitem{inferencia_wiki}
	WIKIPEDIA. Estadística Inferencial,
	\url{https://es.wikipedia.org/wiki/Estadística_inferencial},
	
	\bibitem{analitics}
	SRIVASTAVA, T. Difference between machine learning and statistical modeling.
	\url{https://www.analyticsvidhya.com/blog/2015/07/
		difference-machine-learning-statistical-modeling/}
	
	\bibitem{ISL_python}
	Müller, A. C., and Guido, S. Introduction to Machine Learning with Python:
	A Guide for Data Scientists. O’Reilly Media, Inc., 2016.
	
	\bibitem{cleverdata}
	GONZÁLEZ, A. Conceptos básicos de Machine Learning,
	\url{http://cleverdata.io/conceptos-basicos-machine-learning/}
	
	\bibitem{clustering}
	BUSINESS ANALYTICS. An Introduction to Clustering \& different methods of clustering,
	\url{https://www.analyticsvidhya.com/blog/2016/11/an-introduction-to-clustering-and-different-methods-of-clustering/}
	
	\bibitem{carlosiii}
	GRANÉ, A. Análisis de Componentes Principales,
	\url{http://halweb.uc3m.es/esp/Personal/personas/agrane/ficheros_docencia/MULTIVARIANT/slides_comp_reducido.pdf}
	
	\bibitem{PCA}
	DE LA FUENTE, FERNÁNDEZ, S. Componentes Principales,
	\url{http://www.fuenterrebollo.com/Economicas/ECONOMETRIA/MULTIVARIANTE/ACP/ACP.pdf}
	
	\bibitem{naive}
	MALAGÓN, LUQUE, C. Clasificadores bayesianos. El algoritmo Naive Bayes,
	\url{https://www.nebrija.es/~cmalagon/inco/Apuntes/bayesian_learning.pdf}
	
	\bibitem{decision_trees}
	WIKIPEDIA,  Árboles de decisión,
	\url{https://es.wikipedia.org/wiki/Árbol_de_decisión}
	
	\bibitem{ventajas_trees}
	CUENCA, J.  Ventajas y desventajas. Teoría de decisiones y árboles de decisión.
	\url{http://teoriadedecisionesyarbolesdedecision.blogspot.com/}	
	
	\bibitem{wiki_xgboost}
	WIKIPEDIA. XGBoost.
	\url{https://en.wikipedia.org/wiki/Xgboost}		
	
	\bibitem{XGBoost}
	BROWNLEE, J,  XGBoost With Python. Gradient Boosted Trees with XGBoost and scikit-learn. Machine Learning Mastery. 2016
	
	\bibitem{RLWiki}
	WIKIPEDIA. Logistic Regression.
	\url{https://en.wikipedia.org/wiki/Logistic_regression}	
	
	\bibitem{RLIntro}
	MORAL, PELÁEZ, I.  Modelos de regresión: lineal simple y regresión logística.
	\url{http://www.revistaseden.org/files/14-cap\%2014.pdf}
	
	\bibitem{RL_large}
	DE LA FUENTE, FERNÁNDEZ, S.  Regresión Logística. 2011.
	\url{http://www.estadistica.net/ECONOMETRIA/CUALITATIVAS/LOGISTICA/regresion-logistica.pdf}	
	
	\bibitem{gaussiannb}
	SCIKIT-LEARN.  sklearn.naive\_bayes.GaussianNB.
	\url{http://scikit-learn.org/stable/modules/generated/sklearn.naive_bayes.GaussianNB.html}
	
	\bibitem{logisticregression}
	SCIKIT-LEARN.  sklearn.linear\_model.LogisticRegression.
	\url{http://scikit-learn.org/stable/modules/generated/sklearn.linear_model.LogisticRegression.html}
	
	\bibitem{cv-wiki}
	WIKIPEDIA. Validación Cruzada.
	\url{https://es.wikipedia.org/wiki/Validación\_cruzada}
	
	\bibitem{confusion}
	BROWNLEE, J. What is a Confusion Matrix in Machine Learning.
	\url{https://machinelearningmastery.com/confusion-matrix-machine-learning/}
\end{thebibliography}


