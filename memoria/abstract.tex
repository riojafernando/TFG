\chapter* {Abstract}

In the Intensive Care Unit, an excessive amount of daily alarms. These alarms are a high priority events, since the condition of the patients thus it requires. That is why the alarms reach 80 dB, which causes lack of sleep, stress ... And other disorders added to the rest of hospitalized patients.\\
\vspace*{0.75cm}

Many of these alarms arise because signals are processed separately, that are being monitored and compared with a predefined threshold, which it's usually not to be in the Intensive Care Unit.\\
\vspace*{0.75cm}

These unnecessary interruptions have also been shown to have a negative effect on recovery and length of stay. Only between 2\% and 9\% are really important for patient management. So, in this field, there is a great room for improvement.
\vspace*{0.75cm}

In this context, this Final Degree Project, focuses on using machine learning techniques, specifically decision trees performing an exploratory data analysis, to obtain a good classification. The database consists of 750 patients with 3/4 signals collected from each of them in hospitals in the USA and Europe.
\vspace*{0.75cm}

Finally, the results show that the proposed algorithm improves two other models, naive bayes and logistic regression, leaving future research lines open
to the multiclassification or to try other techniques in the exploratory data analysis.

