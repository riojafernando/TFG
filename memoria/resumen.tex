\chapter* {Resumen}


	En la Unidad de Cuidados Intensivos saltan una cantidad desmedida de alarmas diarias. Estas alarmas son de una prioridad elevada, ya que el estado de los pacientes así lo requiere. Es por ello que las alarmas alcanzan los 80 dB, lo que provoca falta de sueño, estrés... Y otros trastornos añadidos al resto de pacientes hospitalizados.\par
	\vspace*{0.75cm}
	Muchas de estas alarmas surgen debido a que se procesan por separado las señales que están siendo monitorizadas y comparadas con un umbral predefinido, que normalmente no es para estar en la Unidad de Cuidados Intensivos.\par
	\vspace*{0.75cm}
	Estas innecesarias interrupciones también se ha demostrado que tienen un efecto negativo en la recuperación y la duración de la estancia. \par
Tan solo entre el 2\% y el 9\% son realmente importantes para el manejo del paciente. Por lo que, en este campo, existe un gran margen de mejora.\par 
	\vspace*{0.75cm}
	En este contexto, este Trabajo Fin de Grado, se centra en utilizar técnicas de \textit{machine learning}, concretamente árboles de decisión realizando un análisis exploratorio de los datos previo, para obtener una buena clasificación. La base de datos consta de 750 pacientes con 3/4 señales recogidas de cada uno de ellos en hospitales de EEUU y Europa.\\
	\vspace*{0.75cm}
	
	Finalmente los resultados demuestran que el algoritmo planteado mejora a otros dos modelos, \textit{naive bayes} y regresión logística, dejando abiertas las lineas futuras de investigación a la multiclasificación o a probar otras técnicas en el análisis exploratorio de datos.
	
	
